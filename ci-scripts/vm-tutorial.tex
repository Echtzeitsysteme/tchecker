\documentclass{article}
\usepackage{graphicx}
\usepackage{hyperref}
\usepackage{listings}
\usepackage{xcolor}

\definecolor{codegreen}{rgb}{0,0.6,0}
\definecolor{codegray}{rgb}{0.5,0.5,0.5}
\definecolor{codepurple}{rgb}{0.58,0,0.82}
\definecolor{backcolour}{rgb}{0.95,0.95,0.92}

\lstdefinestyle{mystyle}{
    backgroundcolor=\color{backcolour},
    commentstyle=\color{codegreen},
    keywordstyle=\color{magenta},
    numberstyle=\tiny\color{codegray},
    stringstyle=\color{codepurple},
    basicstyle=\ttfamily\footnotesize,
    breakatwhitespace=false,
    breaklines=true,
    captionpos=b,
    keepspaces=true,
    numbers=left,
    numbersep=5pt,
    showspaces=false,
    showstringspaces=false,
    showtabs=false,
    tabsize=2
}

\lstset{style=mystyle}

\title{Tutorial: Using the TChecker VM}
\author{Alexander Lieb}
\date{\today}

\begin{document}

\maketitle

\section{Introduction}
This tutorial provides a step-by-step guide for using the VM created with the GitHub Action \href{https://github.com/Echtzeitsysteme/tchecker/blob/feature/vm/.github/workflows/vm.yml}{tchecker/vm.yml}. The VM includes the \texttt{tchecker} tool and the \texttt{tchecker-webapp}, along with example files. The default username and password for the VM are both \texttt{tchecker}.

\section{Prerequisites}
\begin{itemize}
    \item VirtualBox installed on your system.
    \item The \texttt{tchecker.ova} file.
\end{itemize}

\section{Importing the VM into VirtualBox}
\subsection{Using the VirtualBox GUI}
\begin{enumerate}
    \item Open VirtualBox.
    \item Go to \textbf{File} $\rightarrow$ \textbf{Import Appliance}.
    \item Follow the prompts to import the VM.
    \item Start the VM and log in using the username and password \texttt{tchecker}.
\end{enumerate}

\subsection{Using the Command Line}
\begin{enumerate}
    \item Open a terminal or command prompt.
    \item Navigate to the VirtualBox installation directory:
    \begin{lstlisting}[language=bash]
cd virtualbox-install-dir \end{lstlisting}
    \item Import the VM using the following command:
    \begin{lstlisting}[language=bash]
    .\VBoxManage.exe import path/to/tchecker.ova --vsys 0 --vmname name-of-vm --unit 12 \end{lstlisting}
    \item Start the VM from the VirtualBox GUI or using the command line.
\end{enumerate}

\section{Accessing the Examples}
\label{sec:accessing-the-examples}
All examples are located in the directory \texttt{/home/tchecker/tchecker\_code/examples}. Each example consists of a shell script and a \texttt{.txt} file. The shell script generates a \texttt{.tck} file, which is used as input for \texttt{tchecker}.

\begin{enumerate}
    \item Open a terminal in the VM.
    \item Navigate to the examples directory:
    \begin{lstlisting}[language=bash]
    cd /home/tchecker/tchecker_code/examples \end{lstlisting}
    \item Run the shell script for the desired example:
    \begin{lstlisting}[language=bash]
    ./example_script.sh \end{lstlisting}
    \item Use \texttt{tchecker} to analyze the generated \texttt{.tck} file 
    \item For most tools, the description can be found at (\href{https://github.com/ticktac-project/tchecker/wiki/Using-TChecker}{https://github.com/ticktac-project/tchecker/wiki/Using-TChecker}).
    \item For comparison, the tool tck-compare is used:
    \begin{lstlisting}[language=bash]
    tck-compare -h \end{lstlisting}
\end{enumerate}

\section{Using the TChecker WebApp}
\subsection{Starting the WebApp}
\begin{enumerate}
    \item Open a terminal in the VM.
    \item Navigate to the \texttt{tchecker-webapp} directory:
    \begin{lstlisting}[language=bash]
    cd /home/tchecker/Desktop/webapp \end{lstlisting}
    \item Start the web application:
    \begin{lstlisting}[language=bash]
    source venv/bin/activate 
    ./start.sh \end{lstlisting}
    \item Open a web browser and navigate to \texttt{http://localhost:5173}.
\end{enumerate}

\subsection{Analyzing Examples with the WebApp}
\begin{enumerate}
    \item Follow the steps in Section~\ref{sec:accessing-the-examples} to generate a \texttt{.tck} file.
    \item Upload the \texttt{.tck} file to the web application.
    \item Use the web interface to analyze the model.
\end{enumerate}

\end{document}